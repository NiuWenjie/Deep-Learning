\documentclass[10pt,twocolumn,letterpaper]{article}

\usepackage{cvpr}
\usepackage{times}
\usepackage{epsfig}
\usepackage{picinpar}
\usepackage{amsmath}
\usepackage{amssymb}
\usepackage{booktabs}
\usepackage{float}
\usepackage{graphicx}
\usepackage{subfigure}
\usepackage[breaklinks=true,bookmarks=false,colorlinks,
            linkcolor=red,
            anchorcolor=blue,
            citecolor=green,
            backref=page]{hyperref}

\cvprfinalcopy % *** Uncomment this line for the final submission

\def\cvprPaperID{****} % *** Enter the CVPR Paper ID here
\def\httilde{\mbox{\tt\raisebox{-.5ex}{\symbol{126}}}}

% Pages are numbered in submission mode, and unnumbered in camera-ready
%\ifcvprfinal\pagestyle{empty}\fi
\begin{document}

%%%%%%%%% TITLE
\title{Week One: Introoduction to Deep Learning}
\author{Wenjie Niu\\\ July 1st-7st,2018}


\maketitle
%%%%%%%%% BODY TEXT
\section{Courses in This Specialization}
\begin{itemize}
\item Neural Networks and  Deep Learning
\item Improving Deep Neural Networks: Hyperparemeter tuning, Regularization and Optimization.
\item Structure your Machine Learning project
\item Convolutional Neural Networks
\item Natural Language Processing: Building sequence models
\end{itemize}

\section{Neural Network}
It's a powerful learning algorith inspired by how the brain works. Here are 2 examples to show what is a Neural Networks.
\begin{itemize}
\item Example 1: Single Neural Networks
%%%两个图需要自己画
\item Example 2: Multiple Neural Networks
%%%两个图需要自己画
\end{itemize}

\section{Supervised Learning for Neural Network}
In supervised learning, we are given a data set and already know what our correct output should look like, having the idea that there is a relationship between the input and the output as shown in Table~\ref{Tab:Examples }.\par
Supervised learning is categorized into ``regression" and ``classification" problems. In a regression problem, we are trying to map input variables to some continuous function, predicting the results within the a continuous output. In a classification problem, we are trying to map input variables into discrete categories.\par

\begin{table}[H]
\begin{center}
\caption{Here are some examples of supervised learning}
\begin{tabular}{ccc}
\toprule
Input(x) & Output(y) & Application \\
\midrule
Home features & Price & Real estate \\
Ad, user info & Click on ad?(0/1) & Online advertising \\
Image & Object & Photo tagging \\
Audio & Text transcript & Speech recognition \\
English & Chinese & Language translation \\
Image, Radar info & Other cars position & Autonomous driving \\
\bottomrule
\end{tabular}
\label{Tab:Examples }
\end{center}
\end{table}

There are different types of neural network, for example Convolution Neural Network(CNN) used often for image application and Recurrent Neural Network used for one-dimensional sequence data.\par
Another significant concept is about structured in Fig.~\ref{fig:Structured} and unstructured data in Fig.~\ref{fig:Unstructured}. Structured data refer to things that has a defined meaning such as price, age while unstructured data refers to thing like pixel, raw audio, text.\par


\begin{figure}
\begin{minipage}[t]{0.5\linewidth}
\centering
\includegraphics[width=1.5in]{Structured.png}
\caption{Structured data}
\label{fig:Structured}
\end{minipage}%
\begin{minipage}[t]{0.5\linewidth}
\centering
\includegraphics[width=1.62in]{Unstructured.png}
\caption{Unstructured data}
\label{fig:Unstructured}
\end{minipage}
\end{figure}


%{\small
%\bibliographystyle{ieee}
%\bibliography{egbib}
%}

\end{document}
