\documentclass[a4paper]{article}

\usepackage[english]{babel}
\usepackage[utf8]{inputenc}
\usepackage{fullpage}
\usepackage{amsmath}
\usepackage{graphicx}
\usepackage{url}
\usepackage{subfigure}
\usepackage{tabularx}
\usepackage{indentfirst}
\usepackage[colorinlistoftodos]{todonotes}
\usepackage{hyperref}
\usepackage{amssymb}
\usepackage{outline} 
\usepackage{pmgraph} 
\usepackage[normalem]{ulem}
\usepackage{verbatim}
% \usepackage{minted} % need `-shell-escape' argument for local compile

\title{
    \vspace*{1in}
    \includegraphics[width=2.75in]{zhenglab-logo.png} \\
    \vspace*{1.2in}
    \textbf{\huge Weekly Work Report}
    \vspace{0.2in}
}

\author{Wenjie Niu \\
    \vspace*{0.5in} \\
    \textbf{VISION@OUC} \\
    \vspace*{1in}
}

\date{\today}


\begin{document}

\maketitle
\setcounter{page}{0}
\thispagestyle{empty}
\newpage

This week, I start the Deep Learning course. Although I watched the course and took some notes before, the real and deep understanding is not enough for me. Therefore, it's necesaary to learn again. In this week, I'm learning the introduction of deep learning, the main content is here as following.\par

\section{Content}
\subsection{Courses in This Specialization}
\begin{itemize}
\item Neural Networks and  Deep Learning
\item Improving Deep Neural Networks: Hyperparemeter tuning, Regularization and Optimization.
\item Structure your Machine Learning project
\item Convolutional Neural Networks
\item Natural Language Processing: Building sequence models
\end{itemize}


\subsection{Neural Network}
It's a powerful learning algorith inspired by how the brain works. Here are 2 examples to show what is a Neural Networks.
\begin{itemize}
\item Example 1: Single Neural Networks~\cite{mooc.com}.\par 
Given data about the size of houses in real estate market, and you need to fit a function that will predict their price.The curve of the price is similar to Rectified Linear Unit(ReLU) fuction in Fig.~\ref{fig:ReLU}. While the difference is that the real curve can never be negative and it starts at zero.

\begin{figure}[!htp]
\begin{center}
   \includegraphics[width=0.5\linewidth]{ReLU.png}
\end{center}
   \caption{Rectified Linear Unit(ReLU) fuction}
\label{fig:ReLU}
\end{figure}

%%%两个图需要自己画
\item Example 2: Multiple Neural Networks~\cite{mooc.com}.\par
The price of houses can be affected by other features such as size, numbers of bedrooms, zip code and wealth. The role of the neural network is to predict the price and it will automatically generate the hidden units. We only need to give the input x and the output y as shown in Fig.~\ref{fig:Multiple}.
\end{itemize}

\begin{figure}[!htp]
\begin{center}
   \includegraphics[width=0.5\linewidth]{Multiple.png}
\end{center}
   \caption{Multiple Neural Network~\cite{Coursera.org}}
\label{fig:Multiple}
\end{figure}


\subsection{Supervised Learning for Neural Network}
In supervised learning, we are given a data set and already know what our correct output should look like, having the idea that there is a relationship between the input and the output as shown in Table~\ref{tab:Examples}.\par
Supervised learning is categorized into ``regression" and ``classification" problems. In a regression problem, we are trying to map input variables to some continuous function, predicting the results within the a continuous output. In a classification problem, we are trying to map input variables into discrete categories.\par

\begin{table}[hb]
    \centering
    \caption{Here are some examples of supervised learning}
    \begin{tabular}{|c|c|c|}
    \hline
        Input(x) & Output(y) & Application \\
        \hline 
        Home features & Price & Real estate \\ \hline
		Ad, user info & Click on ad?(0/1) & Online advertising \\ 		\hline
		Image & Object & Photo tagging \\ \hline
		Audio & Text transcript & Speech recognition \\ \hline
		English & Chinese & Language translation \\ \hline
		Image, Radar info & Other cars position & Autonomous driving \\ \hline
    \end{tabular}
    \label{tab:Examples}
\end{table}

There are different types of neural network, for example Convolution Neural Network(CNN) used often for image application and Recurrent Neural Network used for one-dimensional sequence data.\par
Another significant concept is about structured in Fig.~\ref{fig:Structured} and unstructured data in Fig.~\ref{fig:Unstructured}. Structured data refer to things that has a defined meaning such as price, age while unstructured data refers to thing like pixel, raw audio, text.\par

\begin{figure}
\begin{minipage}[t]{0.5\linewidth}
\centering
\includegraphics[width=2.5in]{Structured.png}
\caption{Structured data~\cite{Coursera.org}}
\label{fig:Structured}
\end{minipage}%
\begin{minipage}[t]{0.5\linewidth}
\centering
\includegraphics[width=2.62in]{Unstructured.png}
\caption{Unstructured data~\cite{Coursera.org}}
\label{fig:Unstructured}
\end{minipage}
\end{figure}


\subsection{Why Is Deep Learning Taking off?}
Deep learning is taking off due to a large amount of \emph{data} avilable through the digitization of society, faster \emph{computition} and innovation in the development of neural network \emph{algorithm}.\par
When there are a small training set, the performance depends much on skill at hand engineer features. But it doesn't work when a huge training set. Two things have to be considered to get the high level of performance:
\begin{itemize}
\item Being able to training enough neural network
\item Huge amount of labeld data 
\end{itemize}
In a word, scale drives deep learning progress.


\subsection{Some Important in Test 1}
After the part of courses, there are still some concept not being understood. In this subsectio, I will list to enhance them.
\begin{itemize}
\item Structured and Unstructured data.\par
 Structured: Things have a defined meaning like economic \emph{etc.} While Image of cat is unstructured data.
 \item Increasing the size of a neural network generally does not hurt an algorithm's performance, and it may help significantly; Increasing the traing set size generally does not hurt an algorithm's performance, and it may help significantly.
\end{itemize}

\subsection{Binary Classification}
In a binary classificartion problem, the result is a discrete value output.\par
\textbf{For example: Cat vs. Non-Cat}\par
The goal is to train a classifier that the input is an image represented by a feature vector $x$, and predicts whether the corresponding label is 1 or 0. In this case, whether this is a cat image(1) or a non-cat image(0).\par

\begin{figure}[!htp]
\begin{center}
   \includegraphics[width=1\linewidth]{CattoPixel.png}
\end{center}
   \caption{Here is the figure of cat-to-matrics~\cite{mooc.com}\cite{Coursera.org}.}
   \label{fig:cat}
\end{figure}

An image is store in the computer in three seperate matrics corresponding to the Red, Green, and Blue color channels of the image. The three matrics have the same size as the image, for example, the resolution of the cat is 64 pixels $\times$ 64 pixels, the three matrics(RGB) are 64 $\times$ 64 each as figure.~\ref{fig:cat}.\par
The value in a cell represents the pixel intensity which will be used to create a feature vector of n-dimension. In pattern recognition and machine learning,a feature vector represents an object, in this case, a cat or no cat.\par
To create a feature vector $x$, the pixel intensity values will be "unroll" or "reshape" for each other color. The dimension of the  input feature vector $x$ is $n_x$=64 $\times$ 64 $\times$ 3 = 12288.

\begin{equation*}
x=\begin{bmatrix}
	255\\
    231\\
    42\\
    \vdots \\
    255\\
    134\\
    202\\
    \vdots \\
    255\\
    134\\
    93\\
    \vdots 
\end{bmatrix}
\end{equation*}

\textbf{Notation}\par
The last matric in Eq.~\ref{Eq:1}
\begin{equation}
(x,y) \quad \quad x \in R^{n_x},y \in \{0,1\}
\label{Eq:1}
\end{equation} \par
m training example in Eq.~\ref{Eq:2}
\begin{equation}
\{(x^{(1)},y^{(1)}),(x^{(2)},y^{(2)}),\cdots,(x^{(m)},y^{(m)})\}
\label{Eq:2}
\end{equation}


\subsection{Logistic Regression}
Logistic regression is a learning algorithm used in a supervised learning problem when the output $y$ are all either zero or one. The goal of a logistic regression is to minimize the error between its predictions and training data.\par
Like the example in last subsection, given  an image represented by a feature vector  $x$, the algorithm will evaluate the probability of a cat being in that image like~\ref{Eq:3}
\begin{equation}
Given \quad x, \quad \hat{y}=P(y=1|x),\quad where \quad 0 \leq \hat{y} \leq 1
\label{Eq:3}
\end{equation}
The parameters used in Logistic Regression are:
\begin{itemize}
\item The input features vector: $x \in R^{n_x} $, where $n_x$ is the number of features
\item The traing label: $y \in \{0,1\}$
\item The weights: $\omega \in R^{n_x}$, where $n_x$ is the number of features
\item The threshold: $b \in R$
\item The output: $\hat{y}=\sigma(\omega^T+b)$
\item Sigmoid function: $s=\sigma(\omega^T+b)=\sigma(z)=\frac{1}{1+e^{-z}}$
\end{itemize}\par
$(\omega^T+b)$ is a linear function $(ax+b)$, but since we are looking for a probability constraint between [0.1], the sigmoid function as shown in figure.~\ref{fig:sigmoid} is used.

\begin{figure}[!htp]
\begin{center}
   \includegraphics[width=1\linewidth]{Sigmoid.png}
\end{center}
   \caption{Sigmoid function}
   \label{fig:sigmoid}
\end{figure}
Some observation from the graph:
\begin{itemize}
\item If $z$ is a large positive number, then $\sigma(z)=1$
\item If $z$ is small or large negative number, then $\sigma(z)=0$
\item If $z=0$, then $\sigma(z)=0.5$
\end{itemize}


\section{Progress in this week}
This week I learnt the introduction of deep learning including some new concepts such as neural network, supervised learning, and the reason of deep learning's prosperity. The following are the steps you took:
\begin{description}
\item [Step 1]
Watched the courses clips.
\item[Step 2]
Wathced again and took notes.
\item[Step 3]
Grasped the related pictures and wrote the Latex.
\item[Step 4]
Organized the content and push to the github.
\end{description}


\section{Plan}

\begin{tabular}{rl}
	\textbf{Objective:} & Learn Neural Network and Deep Learning by myself. \\
    %\textbf{Deadline:} & XXXX 
\end{tabular}

\begin{description}
    \item[\normalfont 2018.07.08---2018.05.14] Watch the rest of week two course clips and take the note.
    \item[\normalfont 2018.07.15---2018.07.21] Do so on week three course.
    \item[\normalfont 2018.05.22---2018.05.38] Do so on week tfour course.
\end{description}

% If you don't cite any references, please comment the following two lines
\bibliographystyle{ieee}
\bibliography{WeekOne}

\end{document}